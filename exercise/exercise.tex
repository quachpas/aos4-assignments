\documentclass[answers, 10pt, UKenglish]{exam}
\usepackage[UKenglish]{babel} 
\usepackage[]{fontspec} 
\usepackage[]{amsmath,amssymb}
\usepackage[]{cleveref} 
\usepackage[]{cancel} 
\usepackage[]{lipsum} 
\usepackage[]{csquotes} 
\usepackage[]{isodate} 
\usepackage[]{siunitx}
\usepackage[]{textcomp} 
\usepackage[]{tabularx} 
\usepackage[]{pdflscape} 
\usepackage[]{xcolor} 
\DeclareSIUnit[]{\SIeuro}{\texteuro}
\newcommand{\price}[1]{\SI[round-precision=2,round-mode=places,round-integer-to-decimal]{#1}[]{\SIeuro}}
\newcommand{\weight}[1]{\qty{#1}{\gram}}
\newcommand{\dimensions}[1]{\qtyproduct{#1}{\milli\metre}}
\newcommand{\displaysize}[1]{\qty{#1}{"}}
\newcommand{\capacity}[1]{\qty{#1}{\milli\ampere\hour}}

\begin{document}

\title{AOS4 - Smartphone purchase}
\author{Pascal Quach}
\maketitle
% Context: 
%	(Your parents have decided to buy you a new smartphone, and
%	ask you to choose among a list they prepared for you.)
%	(Preferences information?)
%	(Criteria...)
% 	(You search the product details online, and compile them
%	into a table.)

Your parents have decided to buy you a new smartphone, and ask you to choose
among  a list they have prepared for you. Unfortunately for you, you tend to be
\textbf{very thorough} when choosing your smartphone, but cannot actually change the list
without your parents' approval.

Stuck with a fixed list, you resolve yourself to make the best decision
according to your tastes. Fortunately, you have been studying multi-criteria
decision-making (MCDM), what a coincidence!

After consulting with your parents, they give you the following guidelines:

\begin{enumerate}
	\item \textquote{The price cannot exceed \price{1000}}
	\item \textquote{The camera should not influence your decision}
	\item \textquote{It is preferable if you give a ranking, rather than a choice}
\end{enumerate}

After searching for the products' details online, you compile the information into~\cref{tab:em-1}


\begin{questions}
\begin{solutionorbox}
	{
		\color{red}
		This exercise should be solved in class, as practice for MCDM.
		The familiar
		context enables students to compare their own reasoning to the
		studied decision rules, and facilitates reviewing taken
		decisions. 

		The questions let students identify the types of criteria
		(order/type),
		apply Pareto Dominance, lexicographic/WS criteria, and further
		identify the different elicitation scenarios. 
	}
\end{solutionorbox}

	% fig. Evaluation Matrix n°1
	%	(Introduce context problem)
	%	(Naturally ordered numeric criteria)
	\question
	% 0. Choice: which is the best according to you? Why? What are the trade-offs?
	% 	(Introductory question, to analyse the criteria, and the objectives)
	% (Clarify objectives with preference information)
	
	\begin{parts}
		\part \label{q:1a} According to your preferences, and only using the data
		contained in~\cref{tab:em-1}, make a decision and list your top
		3 choices.

		\part \label{q:1b} For each criteria, answer the following questions, and compare your
		answers in group:
		
		\begin{subparts}
			\subpart \label{q:1bi} What objective have you chosen,
			\textit{e}.\textit{g}.\ $\max$, $\min$?
			
			\subpart \label{q:1bii} If you cannot choose an
			objective, how did you order the values?
			
			\subpart \label{q:1biii} Are there criteria that are
			\textquote{not important}?
		\end{subparts}
	\end{parts}
	
	\begin{solutionorbox}
	{ 
		\color{red} Introductory question, familiarize with the context
		and the data. Identify subjective aspects of the criteria, types 
		of criteria, etc.

		For question~\cref{q:1a}, the idea is to introduce the data to the 
		student, and make sure he has gone through the table. Any preferences
		the student has can serve for further reasoning. Why did he prefer
		certain criteria over others? Which objectives did he choose for each
		criteria?

		In~\cref{q:1b}, we 
	}
	\end{solutionorbox}
	
	\question
	% 1. Pareto Dominance
	%	(Apply Pareto Dominance)
	\begin{solutionorbox}
	\end{solutionorbox}

	\question
	% fig. Evaluation Matrix n°2: changed values so Pareto dominance is not applicable
	%	(non ordered criteria)
	%	(consult with expert/friend, added non ordered criteria, e.g. brand, size, etc.)
	% 2.
	% 	2.a Can you still apply Pareto Dominance? 
	%	Would the decision be the same for someone else ?
	%		(Yes/No. Non-ordered criteria, no.
	%		Choice is personal, you rank the criteria)
	% 	txt. Since the choice is yours, you order the criteria.
	% 	2.b. Order the non ordered criteria
	\begin{solutionorbox}
	\end{solutionorbox}
	
	\question 
	% 3. Solve the problem using an aggregation procedure.
	%	3.a. Rank the criteria
	%		(lexicographic, i.e. rank criteria) 
	%		(this can be done in group, swap the rankings)
	%	3.b. Assign weights to the criteria
	%		(weighted sum, i.e. weight criteria) 
	%		(this can be done in group, swap the rankings)
	% 	(change all objectives to identical ones)
	\begin{solutionorbox}
	\end{solutionorbox}
	
	\question 
	% 4. Do you think the decision makes sense? Explain wrt to the aggregation procedures, and the criteria.
	%	(Introduce common sense, and examine the aggregation/criteria)
	%	(Does the ranking make sense? How do you obtain weights?)
	%	(What "real" consequences does the decision have?)
	\begin{solutionorbox}
	\end{solutionorbox}
	
	\question 
	% fig. Product photos
	% 5. Does the product design influence your choice? Would you change
	% your decision? Why? Discuss the reasons in group
	%	(Introduce missing criteria, product personality)
	%	(Open discussion, introduce another subjective criteria,
	%	and showcase the importance of product design
	%	and how it fits "your personality")
	\begin{solutionorbox}
	\end{solutionorbox}
\end{questions}

\begin{landscape}
\begin{table*}[htpb]
	\centering
	\caption{Smartphones details}
	
	\scriptsize
	\label{tab:em-1}	
	\begin{tabularx}{24.5cm}{|X|X|X|X|X|X|X|X|X|X|X|X|}
		\hline
		Name & Brand & Release Date & Dimensions & Weight & 5G & Display Type & Display Size & Operating System & \qty{3.5}{\milli\meter} Jack & Battery capacity & Price \\%&
		\hline\hline
		Samsung Galaxy A52s 5G & Samsung & \printdate{01/09/2021} & \dimensions{159.9 x 75.1 x 8.4} & \weight{189} & Yes & OLED & \displaysize{6.5} & Android & Yes & \capacity{4500} & \price{349.99}\\%&
		\hline
		Apple iPhone 13 Pro Max & Apple & \printdate{24/09/2021} & \dimensions{160.8 x 78.1 x 7.7} & \weight{240} & Yes & OLED & \displaysize{6.7} & iOS & No & \capacity{4352} & \price{1379}\\%&
		\hline
		Xiaomi 11T Pro & Xiaomi & \printdate{05/10/2021} & \dimensions{164.1 x 76.9 x 8.8} & \weight{204} & Yes & OLED & \displaysize{6.67} & Android & No & \capacity{5000} & \price{412.99}\\%&
		\hline
		Xiaomi 12 Pro & Xiaomi & \printdate{31/12/2021} & \dimensions{163.6 x 74.6 x 8.2} & \weight{204} & Yes & OLED & \displaysize{6.73} & Android & No & \capacity{4600} & \price{758.00}\\%&
		\hline
		Asus Zenfone 9 & Asus & \printdate{15/09/2022} & \dimensions{146.5 x 68.1 x 9.1} & \weight{169} & Yes & OLED & \displaysize{5.9} & Android & Yes & \capacity{4300} & \price{743.89}\\%&
		\hline
		OnePlus 10 Pro & OnePlus & \printdate{13/01/2022} & \dimensions{163 x 73.9 x 8.6} & \weight{201} & Yes & OLED & \displaysize{6.7} & Android & No & \capacity{5000} & \price{724.99}\\%&
		\hline
		Nothing Phone (1) & Nothing & \printdate{16/06/2022} & \dimensions{159.2 x 75.8 x 8.3} & \weight{193.5} & Yes & OLED & \displaysize{6.55} & Android & No & \capacity{4500} & \price{399.00}\\%&
		\hline
		Google Pixel 7 Pro & Google & \printdate{13/10/2022} & \dimensions{162.9 x 76.6 x 8.9} & \weight{212} & Yes & OLED & \displaysize{6.7} & Android & No & \capacity{5000} & \price{812.00}\\%&
		\hline
		Asus ROG Phone 6D Ultimate & Asus & \printdate{07/10/2022} & \dimensions{173 x 77 x 10.4} & \weight{247} & Yes & OLED & \displaysize{6.78} & Android & Yes & \capacity{6000} & \price{1399.00}\\%&
		\hline
		Huawei Mate 50 Pro & Huawei & \printdate{28/09/2022} & \dimensions{162.1 x 75.5 x 8.5} & \weight{205} & No & OLED & \displaysize{6.74} & EMUI & No & \capacity{4700} & \price{1154.99}\\%&
		\hline
		Samsung Galaxy S22 Ultra 5G & Samsung & \printdate{25/02/2022} & \dimensions{163.3 x 77.9 x 8.9} & \weight{228} & Yes & OLED & \displaysize{6.8} & Android & No & \capacity{5000} & \price{928.00}\\%&
		\hline
		Motorola Moto X40 & Motorola & \printdate{22/12/2022} & \dimensions{161.2 x 74 x 8.6} & \weight{199} & Yes & OLED & \displaysize{6.7} & Android & No & \capacity{4600} & \price{465.79}\\%&
		\hline
	\end{tabularx}
\end{table*}
\end{landscape}

\end{document}
